\documentclass[a4paper,12pt]{article}
\usepackage[utf8]{inputenc}
\usepackage{cludein}
\begin{document}
\title{Trabalho de Teoria dos Grafos}
\author{Bruno Ramos \and Madson Araújo \and Nemuel Leal \and Nilton Vasques}
\date{\today}
\maketitle

\section{Árvores Geradoras Mínimas}
Definição: Seja G=(V,E) um grafo não direcionado e conexo, G'=(V,E') é chamado de subgrafo gerador se possuí os mesmos vértices de G. Portanto se tivermos em G' uma árvore, então o subgrafo é uma árvore geradora. 
Quando G é um grafo conexo, em que cada aresta possui um valor ou peso p(e), o peso total da árvore geradora é \[\sum_{e \in E'}p(e)\] onde p(e) é uma função que retorna o peso da aresta \emph{e}. Á árvore geradora mínima é a árvore G' que possui o menor peso total dentre todas as árvores possíveis do grafo G. Podemos enunciar a função para encontrar a árvore geradora mínima como \[\emph{min}\sum_{e\in E'}p(e)\].
A partir dessa noção podemos visualizar que encontrar a árvore geradora mínima não é tão trivial assim. Se propormos uma solução pela força bruta, ou seja, encontrar todas as árvores geradoras e assim então verificar qual a que possui o menor peso total. No pior caso quando temos um grafo completo(em que todos os vértices se ligam uns aos outros) teríamos $n^{n-2}$ árvores geradoras onde n é o número de nós[X], sendo assim teríamos uma solução em tempo exponencial $O(n^n)$ e inviável \nocite{*}.
Diante deste cenário alguns matemáticos elaboram soluções para o problema das Árvores Geradoras Mínimas, se utilizando de heurísticas gulosas para encontrar a solução ótima. No presente artigo abordaremos o Algoritmo de Kruskal e o de Prim, como estudo de caso.

\includesvg[5cm]{/home/niltonvasques}{minimum_spanning_tree_2}

\subsection{Algoritmo de Kruskal}
O algoritmo de Kruskal ....

\subsection{Algoritmo de Prim}
O algoritmo de Prim...

\begin{thebibliography}{9}

\bibitem{nogueira}
	Fernando Nogueira,
	\emph{Problema da Árvore Geradora Mínima}.
	UFJF.
\end{thebibliography}

\end{document}


